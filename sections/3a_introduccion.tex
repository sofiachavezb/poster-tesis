\textcolor{mikuPink}{\textbf{\textquestiondown Qué es RAG?}}
Retrieval-Augmented Generation es xuna técnica que alimenta a los modelos generativos con conocimiento recuperado desde una base de datos.
\separador

\textcolor{mikuPink}{\textbf{Problema Identificado:}} Actualmente existen dos alternativas para usar RAG:
\begin{itemize}
	\item Plataformas cloud no-code: Usables pero sin control de parámetros, funcionan como cajas negras. Ej: NotebookLM, ChatGPT y Nouswise.
	\item Programar: Configurable pero no accesible, requiere tiempo y conocimientos de NLP, IR y desarrollo de software avanzado. 
\end{itemize}
$\Rightarrow$ Usuarios no pueden experimentar con configuraciones óptimas para sus datos y tarea específicas.
\separador

\textcolor{mikuPink}{\textbf{Hipótesis de Investigación}}\\
\textcolor{mikuPink}{\textbf{H1}}: OOP + Patrones de diseño (Strategy, Factory, Composite) permiten abstraer complejidad RAG manteniendo control granular\\
\textcolor{mikuPink}{\textbf{H2}}: Interfaces configurables e interactivas pueden lograr usabilidad industrial (SUS $\geq 70$)