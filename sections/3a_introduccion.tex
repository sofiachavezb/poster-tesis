\textcolor{mikuPink}{\textbf{\textquestiondown Qué es RAG?}}
\st{una forma elegante de enchufar ChatGPT con tus pdfs}\\
Alimentar un modelo de lenguaje con una base de datos de documentos
\separador

\textcolor{mikuPink}{\textbf{Problema Identificado:}} Actualmente existen dos alternativas para usar RAG:
\begin{itemize}
	\item Plataformas cloud no-code: Usables pero sin control de parámetros, funcionan como cajas negras. Ej: NotebookLM, ChatGPT.
	\item Programar: Configurable pero no accesible, requiere tiempo y conocimientos de NLP, IR y desarrollo de software avanzado. 
\end{itemize}
$\Rightarrow$ Usuarios no pueden experimentar con configuraciones óptimas para sus datos y tarea específicas.
\separador
\textbf{Hipótesis de Investigación}\\
\textcolor{mikuPink}{\textbf{H1}}: OOP + Patrones de diseño (Strategy, Factory, Composite) permiten abstraer complejidad RAG manteniendo control granular\\
\textcolor{mikuPink}{\textbf{H2}}: Interfaces configurables e interactivas pueden lograr usabilidad industrial (SUS $\geq 70$)
\separador
\textbf{Objetivo Principal}\\
Diseñar, implementar y evaluar un módulo RAG no-code en DashAI que combine configuración granular con usabilidad industrial